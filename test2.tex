\documentclass[11pt]{article}

\usepackage{tikz}
\usepackage{circuitikz}
\usepackage{amsmath}
\usepackage{amssymb}
\DeclareMathOperator{\Tr}{Tr}

\usepackage{color}
\usepackage{listings}
\usepackage{setspace}
\definecolor{Code}{rgb}{0,0,0}
\definecolor{Decorators}{rgb}{0.5,0.5,0.5}
\definecolor{Numbers}{rgb}{0.5,0,0}
\definecolor{MatchingBrackets}{rgb}{0.25,0.5,0.5}
\definecolor{Keywords}{rgb}{0,0,1}
\definecolor{self}{rgb}{0,0,0}
\definecolor{Strings}{rgb}{0,0.63,0}
\definecolor{Comments}{rgb}{0,0.63,1}
\definecolor{Backquotes}{rgb}{0,0,0}
\definecolor{Classname}{rgb}{0,0,0}
\definecolor{FunctionName}{rgb}{0,0,0}
\definecolor{Operators}{rgb}{0,0,0}
\definecolor{Background}{rgb}{0.98,0.98,0.98}
\lstdefinelanguage{Python}{
numbers=left,
numberstyle=\footnotesize,
numbersep=1em,
xleftmargin=1em,
framextopmargin=2em,
framexbottommargin=2em,
showspaces=false,
showtabs=false,
showstringspaces=false,
frame=l,
tabsize=4,
% Basic
basicstyle=\ttfamily\small\setstretch{1},
backgroundcolor=\color{Background},
% Comments
commentstyle=\color{Comments}\slshape,
% Strings
stringstyle=\color{Strings},
morecomment=[s][\color{Strings}]{"""}{"""},
morecomment=[s][\color{Strings}]{'''}{'''},
% keywords
morekeywords={import,from,class,def,for,while,if,is,in,elif,else,not,and,or,print,break,continue,return,True,False,None,access,as,,del,except,exec,finally,global,import,lambda,pass,print,raise,try,assert},
keywordstyle={\color{Keywords}\bfseries},
% additional keywords
morekeywords={[2]@invariant,pylab,numpy,np,scipy},
keywordstyle={[2]\color{Decorators}\slshape},
emph={self},
emphstyle={\color{self}\slshape},
%
}
\linespread{1.3}

\title{Research Summary}
\author{Cesar Eduardo Garza}

\begin{document}
\maketitle

In our research into Topological Electronic Lattices, we have looked into RLC circuits with the following layout:

\begin{figure}[h!]
	\begin{center}
		\begin{circuitikz}
			\draw (0,0)
			to[R=$-R$] (0, 2)
			to[short] (2, 2)
			to[C=$C$] (2, 0)
			to[short] (0, 0);
			\draw(2, 0)
			to[short](4, 0)
			to[american inductor=$L$] (4, 2)
			to[short] (2, 2)
			to[short] node[ground] {} (2, 0);
			\draw (2, 0)
			to[short] node[draw, circle, fill=white, minimum size = 10pt] {$V_1$} (2,3);
			\draw (0,2)
			to[short](-2, 2);
			\draw (4, 2)
			to[C=$C_c$] (6,2)
			to[american inductor=$L$] (6,0)
			to[short] (8,0)
			to[C=$C$] (8, 2)
			to[short] (6,2);
			\draw (8, 0)
			to[short] (10,0)
			to[R=$R$] (10, 2)
			to[short] (8, 2)
			to[short] node[draw, circle, fill=white, minimum size = 10pt]{$V_2$} (8,3);
			\draw(8,0)
			to[short] node[ground] {} (8,0);
			\draw(10,2)
			to[short] (12,2);
		\end{circuitikz}
	\end{center}
\end{figure}

The two loops separated by the Capacitor $C_c$ will henceforth be referred to as "subcircuits". We then applied Kirchhoff's current and voltage laws to obtain the following matrix:
\[
\begin{pmatrix}
{-i \gamma + \frac{1}{\omega} - \omega (1+\kappa)} & {\omega \kappa} \\
{\omega \kappa} & {i \gamma + \frac{1}{\omega} - \omega (1 + \kappa)}
\end{pmatrix}
\begin{pmatrix}
V_1 \\
V_2
\end{pmatrix}
\]
where $\kappa = \frac{C_c}{C}$, $\gamma = \frac{1}{R} \sqrt{\frac{L}{C}}$, and $\omega = \omega' \sqrt{LC}$

We then coupled an arbitrary amount of these circuits using a capacitor $C_k$ as shown below

\begin{figure}[h!]
	\begin{center}
		\begin{circuitikz}
			\draw(0,0)
			to[american inductor = $L$] (0, 2)
			to[short] (2, 2)
			to[C = $C$] (2, 0)
			to[short] (0, 0);
			\draw(0, 2)
			to[C = $C_c$] (-2, 2);
			\draw (2, 2)
			to[short] node[draw, circle, fill=white, minimum size = 10pt]{$V_2$} (2, 3);
			\draw (2, 2)
			to[short] (4, 2)
			to[R = $R$] (4, 0)
			to[short] node[ground] {} (2, 0);
			\draw (4, 2)
			to[C=$C_k$] (6,2)
			to[R=$-R$] (6, 0)
			to[short] node[ground] {} (8, 0)
			to[C=$C$] (8, 2)
			to[short] (6, 2);
			\draw (8,2)
			to[short] node[draw, circle, fill=white, minimum size=10pt]{$V_3$} (8,3);
			\draw (8,2)
			to[short] (10,2)
			to[american inductor=$L$] (10, 0)
			to[short] (8,0);
			\draw(10,2)
			to[C=$C_c$] (12,2);
		\end{circuitikz}
	\end{center}
\end{figure}

This coupling gave us the following system of equations:
\[
\begin{pmatrix}
a^{*} & b & 0 & 0 & \dots & 0\\
b & d & c & 0 & \dots & 0\\
0 & c & d^{*} & b & \dots & 0\\
0 & 0 & b & d & \dots & 0\\
\vdots & \ddots & \ddots & \ddots & \ddots & \vdots\\
0 & 0 & 0 & \dots & b & a
\end{pmatrix}
\begin{pmatrix}
V_1\\
V_2\\
V_3\\
V_4\\
\vdots\\
V_{2n}
\end{pmatrix}
\]
Where $a = i \gamma + \frac{1}{\omega} - \omega (1 + \kappa)$, $b=\omega \kappa$, $c = \omega \kappa'$, $d = i \gamma + \frac{1}{\omega} - \omega (1 + \kappa + \kappa')$, $\kappa' = \frac{C_k}{C}$, and $x^{*}$ refers to the complex conjugate of $x$.
\\
\\
If we refer to the matrix as $A$, then it is easy to see that $A^T = A$, and the matrix is symmetric. Additionally, it is easy to see that $\Tr{(A)} \in \mathbb{R}$ which, while it is not definitive proof, is a strong indication that eigenvalues of $A$ will be found in complex conjugate pairs.
\\
\\
To evaluate the eigenvalues of $A$ symbolically, sympy was used. The following is the code used to generate the matrices for 4 circuits coupled together:

\begin{lstlisting}[language=Python]
from sympy import *
init_printing(use_unicode = True)

w, k, kappa, g = symbols("w k k' g")


a, b, c, d = symbols('a b c d')
firstLine = [-I*g+1/w-w*(1+k),w*k]
evenLine = [w*k, I*g+1/w-w*(1+k+kappa), w*kappa]
oddLine = [w*kappa, -I*g+1/w-w*(1+k+kappa), w*k]
lastLine = [w*k, I*g+1/w-w*(1+k)]


def generateMatrix(size):
    mat = []
    mat.append([*firstLine, *[0]*2*(size-1)])
    for i in range((size - 1) * 2):
        if i % 2 is 0:
            mat.append([*[0]*i, *evenLine, *[0]*(2*(size-1)-i - 1)])
        else:
            mat.append([*[0]*i, *oddLine, *[0]*(2*(size - 1) - i - 1)])
    mat.append([*[0]*2*(size - 1), *lastLine])
    return Matrix(mat)
 
fourCase = generateMatrix(4)
fourEigen = fourCase.eigenvals()
pprint(simplify(fourEigen))
\end{lstlisting}

From running the above code on cases for $n=2,3,4,5,6$, we determined that eigenvalues follow the following order:
First, due to $A$ being tridiagonal, there will always be the two trivial eigenvalues that follow for all couplings of the circuits, even for $n=1$:
\[
\lambda = -\omega (1+\kappa) + \frac{1}{\omega} \pm \sqrt{-\gamma^2 +\omega^2 \kappa^2}
\]
For cases $n > 1$, there are the following additional eigenvalues:
\[
\lambda = -\omega (1+\kappa + \kappa') + \frac{1}{\omega} \pm \sqrt{-\gamma^2 + \omega^2 (\kappa^2 + \lambda' \kappa \kappa' + \kappa'^2)}
\]
where $\lambda'$ is an as-of-yet undetermined variable. Using the code above, we determined the following values of $\lambda'$ for the following values n:
\begin{center}
	\begin{tabular}{ c | c}
		$n$ & $\lambda'$ \\
		\hline
		$2$ & $0$ \\
		$3$ & $\pm 1$\\
		$4$ & $0$, $\pm \sqrt{2}$\\
		$5$ & $\frac{\pm 1 \pm \sqrt{5}}{2}$\\
		$6$ & $0$, $\pm 1$, $\pm \sqrt{3}$
	\end{tabular}
\end{center}

The appearance of the golden ratio $\phi$ for the case $n=5$ is especially intriguing, as it may point towards a geometric explanation. Additionally, cases of even $n$ will always include $\lambda' = 0$ and cases of odd $n$ will never include $\lambda' = 0$.
\\
\\
It is interesting that $\lambda$ is a quadratic function of $\omega^2$, meaning that if we were to set $\lambda = 0$, we will always be able to explicitly solve for the normal modes of the coupled circuits. Indeed, using the following code, I solved for normal modes given $\lambda'$:

\begin{lstlisting}[language=Python]
from sympy import *
init_printing(use_unicode = True)
w, k, kappa, g = symbols("w k k' g")
a, b, c, d = symbols('a b c d')

trivialCase = [-w*(1+k) +sqrt(-g**2+k**2 * w**2)+1/w, \
 -w*(1+k) - sqrt(-g ** 2+k ** 2*w ** 2)+1/w]

def solver(li):
    output = set()
    for i in li:
        x = solve(i, w)
        output.update(x)
    
    return(output)

def pprinter(li):
    for i in li:
        pprint(simplify(i))

def generateEigenValues(li):
    mat = [*trivialCase]
    base = -w * (1+k) + 1/w
    for i in li:
        mat.append(base + sqrt(-(g ** 2) + \
        (w ** 2) * (k ** 2 + i * k * kappa + kappa ** 2) ))
        mat.append(base - sqrt(-(g ** 2) + \
       (w ** 2) * (k ** 2 + i * k * kappa + kappa ** 2) ))
    
    return mat

genCase = [a, b, c]
genEig = generateEigenValues(genCase)
pprinter(solver(genEig))
\end{lstlisting}

This resulted in generating the normal modes for $A$. The two trivial cases for $\lambda$ generate the following four normal modes:
\[
\omega = \pm \frac{\sqrt{(1+\kappa+\frac{1}{2}\gamma^2) \pm \sqrt{\frac{1}{4}\gamma^4 - \gamma^2 \kappa - \gamma^2 + \kappa^2}}}{2 \sqrt{2\kappa + 1}}
\]

Other cases for $\lambda$ generate the following normal modes:
\[
\omega = \pm \frac{\sqrt{(1 + \kappa + \frac{1}{2}\gamma^2) \pm \sqrt{\frac{1}{4}\gamma^4 - \gamma^2 \kappa - \gamma^2 + \kappa'^2 - \lambda' \kappa \kappa'}}}{2 \sqrt{2\kappa + 1 + \kappa'^2 + \lambda' \kappa \kappa'}}
\]

Currently, we are attempting to find more values of $\lambda'$ for larger and larger $n$ to attempt to find a pattern and determine $\lambda'$ from $n$. This would be invaluable in finding all normal modes of the circuit for arbitrary $n$, as well as taking the limit as $n \to \infty$. Additionally, we are in the process of obtaining the necessary materials to create this circuit and test it. I have also written a numerical solver for the eigenvalues for use when testing, and can use this to compare with Fatemeh's results.
\\
\\
\\
Thank you for your time.

\end{document}
